\documentclass[paper=a4, fontsize=11pt]{article}
\usepackage[hmargin=1.2in, vmargin=0.8in]{geometry}
\usepackage{amsmath,amsfonts,amsthm}
\usepackage{fontspec}
\setmainfont{Carlito} % use lualatex
\usepackage{color}
\definecolor{darkblue}{rgb}{.07,.07,.34}
\definecolor{darkred}{rgb}{.70,.08,.08}
\usepackage[pdfborder=0 0 0, colorlinks=true, citecolor=black, linkcolor=black, urlcolor=darkblue, letterpaper, bookmarksopen=true, bookmarksopenlevel=2, pdfauthor={Jon Dehdari}]{hyperref}
\usepackage{graphicx}
\usepackage{setspace}
\newcommand{\code}[1]{{\color{darkred}\texttt{#1}}}
\usepackage{fancyvrb}
\newenvironment{codeblock}{\VerbatimEnvironment\begin{color}{darkred}\begin{Verbatim}} {\end{Verbatim}\end{color}}

\title{Exercise: Log-linear Models}
\author{\href{https://github.com/jonsafari/lt1}{Language Technology 1}}
\date{}

\begin{document}
\maketitle
\begin{spacing}{1.8}

Please email your solution to \url{langtech1saarlandws1617@gmail.com} with the subject header \textbf{HW: Log-linear}, by \textbf{15:00, December 14, 2016}.


\section{Basics}
In this section:
\begin{align}
{\bf a} &= \langle -2.7, 0.3, 1.4 \rangle \\
{\bf b} &= \langle -4.2, -0.8, 3.3 \rangle \\
{\boldsymbol C} &= \begin{bmatrix} 1.1 & 5.0 & -4.4 \\ 3.1 & -8.5 & 0.1 \end{bmatrix}
%{\boldsymbol D} &= \begin{bmatrix} 2.1 & -8.5 & 0.1 \\ 1.1 & 5.0 & -4.4 \end{bmatrix} 
\end{align}

\begin{enumerate}
	\item Calculate $\bf a \odot b$\,.  Show your work.
	\item Calculate the previous question using Python3 and Numpy.  Copy your code into the homework submission (without attachments).
	\item Calculate $\bf a \cdot b$\,.  Show your work.
	\item Calculate the previous question using Python3/Numpy, as before.
	\item Calculate $\bf C \cdot a$\,.  Show your work.
	\item Calculate the previous question using Python3/Numpy, as before.
	\item Why do you get an error for $\bf a \cdot C$ using Python3/Numpy?
	\item Calculate $\sigma(\bf C_2 \cdot b + a_2)$\,. Show your work. Remember that $\sigma(\cdot)$ here is the logistic function.
	\item Calculate the previous question using Python3/Numpy, as before.
\end{enumerate}

\section{Practice}
Clone the class repo:
\begin{codeblock}
git clone https://github.com/jonsafari/lt1
\end{codeblock}

In the \code{examples/} directory you'll find a file called \code{perceptron.py} .
You can run it semi-interactively, as \code{python3 -i perceptron.py}

\begin{enumerate}
	\item Add comments to this file like there's no tomorrow.  Add comments to all the interesting parts of the code, and all the semi-interesting parts, and all the not-so-interesting parts.  Especially pay attention to the comment-stubs, like \code{\# ...} \ and \ \code{""" ... """} \,.  Follow good comment style, as described in PEP 8\footnote{\url{https://www.python.org/dev/peps/pep-0008}} and Google's Python Style Guide\footnote{\url{https://google.github.io/styleguide/pyguide.html?showone=Comments\#Comments}}.  This includes the \code{Args:} and \code{Returns:} sections where applicable.  Become familiar with Python docstrings.  You'll probably need to Google a lot of things for this task.
	\item Small extra credit: can you improve the accuracy of the perceptron?  Don't spend too much time on this.
	\item You can submit this section of the homework by either \textbf{a)} forking the Git repository and putting the URL to your forked repo in the homework document; or \textbf{b)} attach the code in the email that you send, and state this in the homework document.  Either way, explicitly state which of these two options you chose.
\end{enumerate}


\end{spacing}
\end{document}
